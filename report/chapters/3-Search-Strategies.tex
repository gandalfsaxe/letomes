% !TEX root =../thesis-letomes.tex

\chapter{Search Strategies}

\section{Brute Force "Fan Search"}
\section{Evolution Strategies}
We are basing our machine learning efforts on the Evolution Strategy (ES) algorithm outlined by Salimans et al. \cite{Salimans2017} using ES--it being a gradient estimator--makes sense in optimization scenarios where calculating an exact gradient is expensive. The idea is to sample around a starting point, and use a weighted average fitness score to pick a direction to move each cycle. It is generally applicable to unsupervised problems, such as ours; a reinforcement learning problem. There are several flavors of the concept under the ES umbrella term: Covariance Matrix Adaptation (CMA-ES), Natural Evolution Strategy (NES), and Exponential NES, to name a few. The one used in \cite{Salimans2017} and by extension in our project is NES. Its exact implementation details will follow:

\subsection{Theoretical Advantages and Disadvantages of ES Compared to RL}

\subsection{NES Algorithm}
In formal terms: Let \(F\) be the objective function with parameters \(\theta\). NES is the population from a distribution over parameters \(p_\psi (\theta)\), with hyperparameters \(\psi\). The procedure is then to optimize the expected objective value \(E_{\theta \sim p_\psi} F(\theta)\) by searching for \(\psi\) with stochastic gradient ascent. The gradient steps are taken with the estimator: \cite{Salimans2017}.
\begin{equation}
    \nabla_\psi E_{\theta \sim p_\psi} F(\theta)
    = E_{\theta \sim p_\psi} \{F(\theta)~\nabla_\psi~log~p_\psi(\theta)\}
\end{equation}
In a Reinforcement Learning (RL) problem, \(F\) is the stochastic score returned by our environment (the Earth-Mars-Sun system) with \(\theta\) representing the policy of an agent in the environment (delta-v for our spacecraft's path) and \(\psi\) being the parametrization of that policy (launch parameters: angle, velocity, launch-time). The algorithm mutates the launch parameters \(\psi\), checks how the resulting path \(\theta\) performs in the environment \(F\) and moves around the resulting n-dimensional optimization space with gradient descent along its estimated gradient, defined by a normally distributed weighted point cloud \(\epsilon\).
