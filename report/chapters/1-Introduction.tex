% !TEX root =../thesis-letomes.tex

\chapter{Introduction}
\todo{Make this less wall-of-text'y}
Colonization and proliferation has long been an ambition for humanity, and with the earth fully explored (more or less), the next obvious frontier is space. Particularly our moon, and the relatively close and temperate Mars. In a lunar or Martian colony scenario, the first several decades will certainly be completely reliant on continuous support from earth, ferrying supplies and humans between the bodies. However, transporting mass into space is very costly, doubly so if it is to arrive unmolested. Particularly costly is the fact that once one has arrived at the destination, with all that kinetic energy from escaping earth's gravity well, one must turn the rocket around and spend an equivalent amount of fuel to slow down to match the speed of the target body, to avoid simply being gravity slingshot onwards into the void. This implies bringing fuel, which must of course be carried, demanding yet more fuel. This means that small savings in energy at this late portion of the mission effect \texttt{significant} savings in the amount of fuel needed overall. Those savings can be achieved through fly-bys of celestial bodies, each passing pulling and slowing the spacecraft a little bit. That is the crux of this project's existence. In a colonial scenario, we can reasonably sacrifice a great deal of expedience for the hypothesized supply missions, since, if we are sending one each month, and they take two years to arrive instead of three months, the increased flight time will need to be felt only once, whereafter they will be arriving at a steady pace, once per month. Thus, if we can achieve even slight fuel efficiency gains, those much longer trajectories become exceedingly tenable. Finding these Low Energy Transfer Orbits (LETOs) is the topic of this project. More specifically, exploring a certain method for finding them. Evolution Strategies (ES) are a class of machine learning algorithms that attempt to emulate a sort of natural selection process to optimize some black box through an estimated, and not explicitly calculated gradient. We have built a reduced N-body problem solver that takes a set of launch parameters as input, and spits out a trajectory, together with its resultant $\Delta v$. This simulator is our black box. We ultimately reach the conclusion that ES (at least a more or less 'pure' implementation) is not a good fit for the problem space, fraught as it is with local minima and large plateaus. Especially given that the true, non-trivial solutions to the problem are very unique, and the system is quite chaotic, at least for the purposes of ES, which we want to both converge quickly and precisely.

The project is partly based on another with a shared author, but we have reimplemented the work present in that project, and have gone to some length to engineer the simulator for extensibility and maintainability. We confront the often present tradeoff between maintainability and performance, in whose pursuit the simulator is implemented in C, and parallelized for GPUs through CUDA. 

\section{Aim of This Thesis}
\todo{The "Aim of this thesis" list needs more work}
\begin{itemize}
	\item We will now search the for LETOs using evolutionary techniques instead of random. Incidentally a paper \cite{Izzo2018} was published on this exact topic after we started work on this thesis, so it appears to be an idea ripe for exploration.
	\item We will now attempt LETOs to Mars in addition to the Moon.
	\item We will now work with a more realistic model instead of the R3B model. Now that we're doing interplanetary transfers, we will work with patched conic approximations on the analytical side and external libraries on the software side.
\end{itemize}